\section{Fit}

\subsection{Signal Variables}
Let $s$ represent the signal strength, and define coefficients $\{f_{had}^{i}\}$ and $\{f_{\mu}^{i}\}$
such that $s^{i} = f_{had}^{i} \times s$ (resp. $s_{\mu}^{i} = f_{\mu}^{i} \times s$) represent the expected signal yields in each bin $i$ of $H_{T}$ of the
hadronic (resp. muon control) samples.  For each SUSY model considered, the $f$'s are taken from simulated events, and $s$ is the parameter of interest in the fit.

\subsection{Likelihood}
There are three possibilities.

\subsubsection{Full Method}
The likelihood function used is simply the product of the likelihood functions described in the previous sections:
\begin{equation}
L = L_{hadronic} \times L_{mu} \times L_{ph} \times L_{constraint}
\end{equation}
The variables $A$, $k$, $\{Z_{inv}^{i}\}$, $\{ttW^{i}\}$, as well as the ``systematic'' variables $\rho_{phZ}$, $\rho_{muW}$, $\rho_{sig}$, $\{\tau_i\}$,
are treated as nuisance parameters.

\subsubsection{Total Background Method Only}
The non-hadronic likelihood functions are simply dropped:
\begin{equation}
L = L_{hadronic}
\end{equation}
The variables $A$, $k$, as well as the ``systematic'' variables $\rho_{sig}$, $\{\tau_i\}$, are treated as nuisance parameters.

\subsubsection{Electroweak Background Constraints Only; Assume QCD-free}
The likelihood function used is simply the product of the likelihood functions described in the previous sections:
\begin{equation}
L = L_{hadronic} \times L_{mu} \times L_{ph}
\end{equation}
with these modifications:
\begin{itemize}
\item $L_{constraint}$ is not used and $QCD^i$ is removed from the problem.
\item $b^i$ is defined as $Z_{inv}^{i} + (t\bar{t}^i+W^i)$.
\item The $\tau^i$ pieces are dropped from $L_{hadronic}$.
\end{itemize}
The variables $\{Z_{inv}^{i}\}$, $\{ttW^{i}\}$, as well as the ``systematic'' variables $\rho_{phZ}$, $\rho_{muW}$, $\rho_{sig}$,
are treated as nuisance parameters.

